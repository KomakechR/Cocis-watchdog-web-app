\documentclass{article}
\begin{document}

\title{COCIS WATCHDOG WEB APPLICATION}
\author{GROUP 21}

\maketitle

\section{Introduction}

In this internet age, access to information has become a fundamental human right especially in developed countries where the well-being of citizens is put at the forefront. It is therefore unfortunate that in developing countries the process to information access has taken on a rather slower path. With emphasis focused around College of Computing and Information Sciences (COCIS), Cocis Watchdog is a web platform developed specifically to keep students up to date with the activities taking place at the college by constantly providing information about; dates for coming tests, dates of upcoming events to be organized at the college, results from college elections, internship as well as job opportunities etc.

\subsection{Background}
Every big organization needs a communication platform to ensure the smooth running of its internal activities take for example;
With this view in mind, and zeroing it down to our faculty COCIS, there is lack of a formal platform that can be used by the administrators as well as the students to effectively pass information to each other. The administrators have hence resorted to pinning posters on notice boards as well as on the college entrance in the hope that the information reaches all concerned personnel or at least those it reaches get to inform the rest.This has not stopped students who fall victim of missing out on crucial announcements casting the blame on the administrators for lack of proper communication as they make the claim of having to work first before they appear for lectures in the evening.

\subsection{Problem Statement}
At COCIS, a college comprised of two schools namely; School of computing and informatics Techology(SCIT) and East African School of Library and Information Studies(EASLIS), there is generally an alarming level of turn up for events organised at the faculty given that both schools have a combined total of approximately one thousand five hundred students. To get to the core of why this is so, we carried out a survey using questionnoires distributed to students at the College. From our review of the questionnoires, the most predomiant reason for the low turn up was due to the fact that most students were unaware as to when such events occured. On the side of the organisers, they were frustrated by the low turn ups yet alot of money is invested in the organisation process in form of printing posters. Looking at the views of both parties involved, it all comes down to one thing; lack of a form platform to use for deliverig information to the students.
Another case that has been presented by students on various occasions is missing tests usually scheduled by lecturers. Since announcement of such tests is done in class, there is bound to be miscommunication hence some students end up having the wrong date for the test.This in turn results in the dreaded act of pleading to the lecturer to be considerate and grant the culprits the test; another point in case of poor communication.
As is with the case of final examinations where the timetable is accessible and downloadable online, the COCIS Watchdog web platform will ensure that information about each and every event at the college is posted on the platform and updated as frequently as possible.
    

\subsection{Objectives}

\subsubsection{General Objective}
The long term goal of the research is to develop a formal platform accessible by all students; that will ensure efficient and timely circulation of information in regards to major announcements about upcoming activities at the college. 
\subsubsection{Specific Objectives}
To combat littering at the college by encouraging use of the web platform as opposed to paper-based posters.\newline
To offer an equal opportunity to all events at the college as far as publicity is concerned without favor\newline
To ensure that students’ concerns such as complaints over missing marks, poor internet connectivity etc reach the college administrators.\newline
To make students more informed about their college by making information such as their respective heads of departments available on the platform, history of the college, pioneers of the college etc.\newline
To provide students with all the necessary study documents such timetables, notes etc all on the same platform.

\subsection{Scope}
Being the initial version, the web application will be implemented to cover activities that are carried out around the college of computing and information sciences (COCIS). The web platform will be accessible by any online users though the information contained on the site will generally be about events at COCIS.

\subsection{Significance of the study}
The study will create a formal communication platform at the college that will be used by both the administrators as well as the students fraternity to pass across any urgent information as regards;
Key events to take place at the college.
Copy of examination, test and class timetables.
Announcement of results from college elections.
Heads of departments of the different courses. etc.  
\newpage
\section{Literature Review}
\subsection{Introduction}
This section addresses the existing systems and their limitations as far as meeting user needs is concerned. Based on these limitations as identified for each reviewd system, we intend to develop a web application that overcomes limitations as will be described in the concluding remarks of this section. Among the notable existing web applications that we considered for review include;

\subsection{Makerere University E-Learning Environment (MUELE)}
This is a website that was developed by Makerere University to act as a platform where students could easily access academic related information such as;\newline lecture slides posted by lecturers of respective courses.\newline Ongoing activities at the university such as seminars.\paragraph
Despite the many benefits that come with having a central platform where every student can access information about the university, it also comes with a number of challenges as outlined below;\newline
Due to the earlier desire by the developers of the MUELE platform to cover the whole university, it increased the workload on the person who would be tasked with updating the platform and this is evident by the course unit under some courses having outdated notes.\newline Some important functionalities such as the ability of students to make complaints about missing marks to lecturers directly instead of filling in paper forms were foregone during the development of the platform.
\subsection{ Facebook}
Facebook is a web application that was founded in 2004 \cite{r3} with the aim of logical connecting geographically distant people.   



\newpage
\section{Methodology}
\subsection{Introduction}
This section highlights the different tools and techniques that will be used to gather information, design the web application and implement it. It will also provide a brief guide on how the system will be tested and validated.

\subsection{Data Collection}
This will be accomplished through document review and interviews with students at COCIS. The tools to be used include audio recorders and questionnaires.
\subsubsection{Questionnaires}
Questionnaires will be designed and distributed among students at the college to collect the necessary information especially in times when interviews are not conducted due to the limited time allocated for the research.

\subsubsection{Document Review}
This will involve reading existing documentation on already developed systems. This is through reading journals, searching the internet etc.

\subsubsection{Oral Interviews}
One on one discussions will be held with a crop of randomly selected students at COCIS who are expected to interact with the system. This will be done in a bid to validate the information gathered through other data collection methods.

\subsection{System Design}
Among the many logical modelling techniques available on the market such as use case diagrams, functional decomposition diagrams etc., we agreed to use Data Flow Diagram modelling technique to design the logical model of the web application. This is due to its many varied advantages over other logical modelling techniques. 

\subsection{System Implementation}
The implementation of the web application will involve the use of various technologies such as client-side and server-side scripting using programming languages like PHP, MySQL, CSS, HTML, JavaScript.



\newpage
\nocite{*}
\bibliographystyle{IEEEtran}
\bibliography{bibliography}

\newpage
\section{Appendix}
\end{document}
